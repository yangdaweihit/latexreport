% 指明编写文档时使用的编码格式:GBK或UTF8。推荐UTF8。
\documentclass[UTF8]{article} 
\usepackage{indentfirst}
\usepackage{xeCJK}
\setCJKmainfont{FandolSong} %中文有衬线字体
\setCJKsansfont{FandolHei}  %中文无衬线字体
\setmainfont{Times New Roman}  %英文有衬线字体
\setsansfont{Fira Sans}  %英文无衬线字体

\title{xeCJK}
\date{}
\begin{document}
\maketitle

\texttt{xeCJK}是用于中文和英文字体单独设置的宏包。

\section{字体示例}

\textrm{有衬线字体}

\textrm{abcdefg}

\textsf{无衬线字体}

\textsf{abcdefg}

\section{用法}
\begin{verbatim}    
\setCJKmainfont{FandolSong} %中文有衬线字体
\setCJKsansfont{FandolHei}  %中文无衬线字体
\setmainfont{Times New Roman}  %英文有衬线字体
\setsansfont{Fira Sans}  %英文无衬线字体
\end{verbatim}    

\section{字体名}
系统内具体有哪些字体需要事先查明,在\texttt{Linux}中终端查找字体名命令如
下:

\begin{verbatim}
fc-list > fontlist.txt #列表冒号前的为字体族名
fc-list -f "%{family}\n" :lang=zh > zhfont.txt #中文字体族名
\end{verbatim}

\end{document}
%%% Local Variables:
%%% mode: latex 
%%% TeX-master: t
%%% End:
