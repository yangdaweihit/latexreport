\documentclass{hitec}
\usepackage[UTF8]{ctex}
\usepackage[dvipsnames]{xcolor}
\usepackage{amsmath}
\usepackage{array}
\usepackage{longtable}
\usepackage{booktabs}  %用于美观表格线宽
\usepackage{multirow}  %用于跨行表
\usepackage{tabularx}  %定义表格宽度
\usepackage{float}
\usepackage[linkcolor=gray!40!black]{hyperref}
\usepackage{tcolorbox}
\usepackage{xcolor}
\tcbuselibrary{breakable,listings}
\graphicspath{{figure/}}

\newtcblisting{code}{%
breakable,
left=3mm,
listing only,
boxrule=0.2mm,
colback=gray!5,
coltext=black
}%

\newtcblisting{cmd}{%
breakable,
left=3mm,
listing only,
boxrule=0.2mm,
colback=black,
coltext=green
}%

\newtcblisting{codeout}{%
breakable,
left=3mm,
boxrule=0.2mm,
colback=gray!5,
coltext=black
}%

% \newtcbox{\verbbox}{on line,
% arc=0pt,outer arc=0pt,colback=grey,colframe=black,
% boxsep=0pt,left=1pt,right=1pt,top=1pt,bottom=1pt,
% boxrule=0pt,bottomrule=1pt,toprule=1pt}
\newtcbox{verbbox}{on line,
arc=2.7pt,outer arc=3pt,colback=gray!5!white,colframe=gray!50!white,boxsep=0pt,left=1pt,right=1pt,top=1pt,bottom=1pt,boxrule=0.3pt,listing only}
\newcommand{\integralx}[3]{\int_{#1}^{#2} #3 \mathrm{d}x }
\newcommand{\mafan}[3]{\sqrt{\sqrt{{#1}^2+{#2}^2}+{#3}^{4}}}
%%% Local Variables:
%%% mode: latex
%%% TeX-master: "latexwritting"
%%% End:

\title{使用\LaTeX{}撰写报告}
\author{杨大伟}
\company{陈少峰-李大军团队}
\confidential{\textbf{-- 非限制发布 --}}
\usepackage{hyperref} % This line is readily ommited of it makes trouble
\begin{document}
\maketitle
%\begin{abstract}
%本文说明如何用\LaTeX{}模板撰写报告。
%
%\textbf{关键词:}线性模型;系数插值
%\end{abstract}
本文的最新版本见于\url{https://github.com/yangdaweihit/latexreport}。

\tableofcontents
\section{前言}
随着我们的讨论日渐深入,论述角度和结论也变得多样和细致。如果我们现在回想一下所经历的论证过程,可能已经难以列举出所有细节。我们的研究是猜想后求证,一旦在某个细节上存在偏差,那么就要回到前一个已验证节点处重新做出判断,再继续做验证。为了能够清晰记录我们所经历的研究历程,需要做出及时和准确的报告,这样才能保证我们的研究方式不会随着研究历程的变长而变得低效。现在我恢复了曾经使用过的\LaTeX{}模板\texttt{hitec.cls}用于我们日常的报告。该模板来自于\texttt{Eli Billauer(2001)}(\url{eli_billauer@yahoo.com})。

为了平衡我们可怜的学习成本和所追求的高效率,在未来一段时间内,我会用以\LaTeX{}的形式撰写报告,同时让小鲍完善细节。各位研究生随后以我报告为示例,撰写自己的报告。在这段过渡时间内,大家可以同时使用\texttt{Word}模板和\LaTeX{}模板,视自己的工作效率而定。但在11月之后需全部转为\LaTeX{}的写作方式。

本文帮助大家度过使用\LaTeX{}最初比较痛苦的阶段,可以使您熟悉最基本的
\LaTeX{}命令和环境,使用这个模板做出高效和高质量的研究报告。当然,本文
假定你已经看过某人在你面前演示过\LaTeX{}的编译过程,教会了你如何安装\texttt{texlive}和如何使用某款撰写和编译\LaTeX{}的集成环境,如\texttt{TeXMaker}。

本文包括\LaTeX{}基本命令和环境、自定义命令和环境、自定义符号的方法。为了实现合作写作,我们还做了一些关于文件组织的约定。在模板的使用过程中,我们会发现一些实际需求,比如老师的批注、对其批注的引用等,对于需求我们将定义一些命令或环境,这样有助于我们提升交流的质量。在\LaTeX{}的介绍过程中,文中还提供了相关内容的来源。追溯这些来源,我们就可以慢慢详细了解\LaTeX{}知识。

\section{\TeX{}和\LaTeX{}的历史}
在不了解关于\TeX{}和\LaTeX{}的最基本历史之前可以学习它们吗?当然可以。但如果你的好奇心带给你太多疑问,比如:这东西看起来怎么这么怪?它怎么发音呀?它到底是什么?我为什么要学习它?那最好通过阅读尽快解决这个问题,这里提供两个链接:

\begin{itemize}
\item 中文介绍:\url{http://www.ctex.org/documents/shredder/tex_frame.html}
\item 英文介绍:\url{https://en.wikipedia.org/wiki/LaTeX}
\end{itemize}

第一次就能读准它很重要:\TeX{}——“泰赫”和\LaTeX{}——“雷泰赫”。

\section{自学资料}
如果你没兴趣看本文的解释,情愿自己去学习权威教程或获取更丰富的资源。那就给你推荐几个好去处。

\subsection{入门教程}
\begin{itemize}
\item 维基教程:\url{https://en.wikibooks.org/wiki/LaTeX}
\item TUG官网教程:\url{http://www.tug.org/twg/mactex/tutorials}
\item \LaTeX{} for Complete Novices: 本地文档,执行命令打开该文档。 
  \begin{cmd}
texdoc dickimaw-novices
  \end{cmd}
 \item \LaTeXe{}: An unofficial reference manual
  \begin{cmd}
texdoc latex2e
 \end{cmd}
 \item The Not So Short Introduction to \LaTeXe{}
   \begin{cmd}  
texdoc lshort
   \end{cmd}
\item 刘海洋. \LaTeX{}入门,电子工业出版社,2013
\end{itemize}

\subsection{网站}

\begin{itemize}
\item TUG: \url{http://www.tug.org},\TeX{}用户群(\TeX Users Group, TUG)网站,全世界\TeX{}用户的组织。这里可以下载到\texttt{TeXLive CD}。
\item CTAN: \url{http://www.ctan.org},(Comprehensive TeX Archive Network,CTAN),你想要的所有文档、模板都在这里。再强调一下,是所有的都在这里。
\item LaTeX: \url{http://www.latex-project.org/},\LaTeX{}官方网站。
\item Stackexchange:\url{http://stackexchange.com},专业的技术交流网站。 
\item \LaTeX{}工作室:\url{http://www.latexstudio.cn},国内稀有高水准排版服务。
\end{itemize}


\section{文档结构}
用\LaTeX{}写作要经历如下过程:
\begin{itemize}
\item 撰写\texttt{tex}文档;
\item 编译文档并转为特定的阅读格式,一般是\texttt{pdf}文档。
\end{itemize}

我们要做的其实只是撰写文档,其余部分都会通过简单的命令由程序完成。首先介绍\texttt{tex}文档的基本结构。\LaTeX{}文档包括\emph{导言}和\emph{主体}两部分。

导言部分用于声明文档的类型和引用的宏包。文档类型决定了文档编译后所套用的模板、为文档撰写准备的一些变量设置,本文所套用的模板就是\texttt{hitec},作者说要做一个简单的用于\emph{高科技}写作的模板。

主体结构由一个环境构成\verb!\begin{document} ... \end{document}!。在这两个命令中间便是你撰写文档内容的地方。文档中\verb!%!之后本行内容不被编译,换言之,\verb!%!是\emph{注释符}。所有\LaTeX{}命令都是用\verb!\!开头的,它的中文名字叫\emph{反斜杠}。

  
\begin{code}
% -----------------导言区
\documentclass{hitec}
\usepackage[UTF8]{ctex}  %引用宏包并加注选项
\usepackage{amsmath}
\usepackage{tcolorbox}
\tcbuselibrary{listings}
% -----------------主体
\begin{document}
 %您的大作
\end{document}
\end{code}
当你写完了文档,就可以编译文档了,比如文档名为\texttt{main.tex}。如果在命令行中操作,是这样的:
\begin{cmd}
 > latex main.tex 
\end{cmd}

但我们多数文档会含有中文,此时应使用\texttt{xelatex}编译:
\begin{cmd}
 > xelatex main.tex
\end{cmd}

然而实际上我们很少在命令行中输入命令编译,多数是在一个集成环境中用快捷完成编译。比如在\texttt{TeXMaker}中的\texttt{F1}键,在\texttt{Emacs AUCTex}中的\texttt{C-c C-c}。总之,在撰写\LaTeX{}文档时,有一款高效的软件是非常必要的。

\section{基本命令和环境}
命令和环境驱动了整个文档的编译。命令的基本基本结构是\verb!\<命令名>[选项]{参数}!。环境是由一对命令构成的---\verb!\begin{环境名}[选项]!和\verb!{\end{环境名}!。\verb![选项]!分为必选和可选两类,由具体命令和环境的定义决定。

\subsection{题目、作者和日期}

文章的开头首先是题目和作者、日期。这三个信息都以命令形式定义的。它们都需要在导言区定义,然后在正文中用一条命令\verb!\maketitle!将它们显示出来。

\begin{code}
...
\title{都闪开,我要用\LaTeX{}写作了}
\author{西门吹牛}
\date{2015年某月某日}
...
\begin{document}
 \maketitle  % 没有这条命令,前面的定义都烟消云散。
 ...
\end{document}
\end{code}

\subsection{中文输入}
在导言区只要引用了宏包\verb!ctex!,即可用命令\verb!xelatex!编译中文了。注意,一般要在选项中说明使用编码\verb!UTF8!。
\begin{code}
\usepackage[UTF8]{ctex}
\end{code}

\subsection{段落}

分段有两种方式:
\begin{itemize}
\item 在段落后填加\verb!\\!
\item 在段落后填加一个空行。
\end{itemize}

\subsection{章节}

章节命令能用到的就三个:
\begin{itemize}
\item 章:\verb!\chapter{绪论}!,在我们这个模板中这一级题目是无效的。
\item 节:\verb!\section{前言}!
\item 次节:\verb!\subsection{背景}!
\end{itemize}


\subsection{列举}

\begin{codeout}
\begin{itemize}
\item 首先,我要强调。
\item 其次,我还要说明。
\end{itemize}
\end{codeout}


\subsection{字体}

我们在写作时为了强调某些内容,或纯粹为了避免单调,会适度地转变字体。在排版中,字体其实是一个含义丰富的术语。当我们在说一种字体时,实际上是一个属性集合,它包括:
\begin{codeout}
\begin{description}  
\item[大小] 比如:{\zihao{3}字号三}、{\zihao{5}字号五}、{\zihao{7}字号七}。
\item[字形] 比如:粗体{\textbf{bfseries}}、斜体{\textsl{slant}}、
打印体{\texttt{italic}}。
\item[字族] 比如:\songti{宋体}、\heiti{黑体}、\fangsong{仿宋}。
\end{description}  
\end{codeout}

其它还有一个属性“字族”,因很少涉及,这里暂略去。关于字体还有更丰富的命令,暂时也不详谈。

改变字体一般有两种方式:一种是改变所选择的文字,称为“区域命令”;
另一种是改变命令声明以后的字体,称为“声明命令”或“模式命令”。

也许你会在有的地方看过命令,如\verb!\bf!,\verb!\md!,\verb!it!,
\verb!\sl!,\verb!\sc!,\verb!\sf!,\verb!\tt!,或\verb!\rm!等。这些命
令已经是陈旧命令,不提倡再使用。原因这些命令不允许复合作用,如:
\begin{codeout}
  \it{\bf{ it and bf format}}
\end{codeout}
显然,\verb!\it!命令已经被\verb!\bf!屏蔽了。1994年发布了\LaTeXe{}取代
了老版本的\LaTeX{}的2.09版,使用了新的命令风格\verb!\textxx!,如:
\begin{codeout}
\textit{\textbf{Never mind the rain}}
\end{codeout}

改进后的字体命令就可以嵌套作用了。

字体声明命令见表\ref{tbl:fontdeclaration}。字体声明命令还可以作为环境
使用。表中命令嵌套使用对临时设置
字体时很重要。

\begin{table}[H]
  \centering
  \caption{字体改变声明}
  \label{tbl:fontdeclaration}
  \begin{tabular}{lll}
   \toprule
   \textbf{Declaration} & \textbf{Example Input} & \textbf{Corresponding output} \\
   \midrule   
   \verb+\rmfamily+ & \verb+\rmfamily roman text+ & {\rmfamily roman text }\\
   \verb+\sffamily+ & \verb+\sffamily sans serif text+ & {\sffamily
                                                         sans serif text }\\
   \verb+\ttfamily+ & \verb+\ttfamily typewritter text+ & {\ttfamily
                                                          typewritter text}\\
   \verb+\mdseries+ & \verb+\mdseries medium text+ & {\mdseries
                                                     medium text}\\
   \verb+\upshape+ & \verb+\upshape upright text+ & {\upshape
                                                     upright text}\\
   \verb+\itshape+ & \verb+\itshape italic text+ & {\itshape
                                                     italic text}\\
   \verb+\slshape+ & \verb+\slshape slant text+ & {\slshape
                                                     slant text}\\
   \verb+\em+ & \verb+\em emphasized text+ & {\em emphasized text} \\
   \verb+\normalfont+ & \verb+\normalfont default text+ & {\normalfont
                                                     default font}\\
   \bottomrule
  \end{tabular}
\end{table}

\subsection{颜色}
\url{https://www.sharelatex.com/learn/Using_colours_in_LaTeX}

\begin{codeout}
{\color{red} 红色}\par
\noindent
{\color{blue} \rule{\linewidth}{0.5mm}}

\colorbox{BurntOrange}{颜色}
\end{codeout}
\subsection{特殊字符}
在文档写作中,经常涉及希腊字母或其它特殊字符的输入。\LaTeX{}已经状备好了这些字符,只需要通过命令实即可实现输入。

\begin{codeout}
%\usepackage{amsmath}
$\alpha$ $\beta$ $\xi$ $\sigma$ $\theta$
\end{codeout}

这些字符已经汇总到了一个文档中供查询—\texttt{symbols-a4.pdf}。在字符界面中,输入如下命令即可自动找到该文档并用系统默认的pdf浏览器打开:
\begin{cmd}
 > texdoc symbols-a4
\end{cmd}

还有一些网站专门提供了手写识别\LaTeX{}符号的功能,只需要用鼠标画出查询的符号,网站就会推荐出可能的符号命令。这里推荐一个:

\url{http://detexify.kirelabs.org/classify.html}


\subsection{数学公式}

参考阅读:\url{https://en.wikibooks.org/wiki/LaTeX/Mathematics}

\begin{codeout}
%\usepackage{amsmath}
一行文字之间的公式叫行内公式,比如:$x^2=1$。行间公式用公式环境,比如:
\begin{eqnarray}
\int_{0}^{1} x^2 \mathrm{d}x & = & \dfrac{1}{3}  \\
\int_{0}^{1} x^2 \mathrm{d}x & = & \sqrt{ \dfrac{1}{3}} \\
\integralx{0}{1}{x}  & = & 23 \\
\mafan{a}{\zeta}{\zeta} &  & \\
\mafan{\gamma}{\delta}{\xi} & = & \\
\end{eqnarray}
式中:注意$\mathrm{d}$是正体,表示微分符号。
\end{codeout}

\subsection{表格}
参考阅读:\url{https://en.wikibooks.org/wiki/LaTeX/Tables}

表格和后面要讲的插图,在\LaTeX{}中称为\texttt{浮动体}。各类表格环境声明于浮动体\texttt{table}之中。浮动体的位置通过选项为控制,一般我们为了使表格处于期望的当前位置,需要引用一个宏包\texttt{float},然后使用选项\texttt{[H]}。要引用表格,需要在浮动体中用命令\verb!\label!声明该浮动体的名字,再用\verb!\ref!引用即可。

\begin{codeout}
\begin{table}[H]
\caption{tabular示例}
\label{table:first}
\begin{center}
%\begin{tabular}{rcl}
\begin{tabular}{>{\raggedleft}p{0.2\linewidth}
                >{\centering}p{0.2\linewidth}
                p{0.2\linewidth}}
\toprule
 左对齐列 & 中间对齐列 & 右对齐列 \\
 \hline
 abcde  & o & edcba \\
 123   & 0 & 321 \\
\bottomrule
\end{tabular}
\end{center}
\end{table}

计算结果见表\ref{table:first}。

\end{codeout}

\LaTeX{}中的表格仅有\texttt{tabbing}和\texttt{tabluar}两种环境,为了丰富和提升表格的功能,大量表格类宏包被开发出来,如可定制宽度的\texttt{tabularx}、可跨页的\texttt{longtable}、具有更多丰富功能的\texttt{tabu}、以及彩色盒子\texttt{tcolorbox}等。本文中的所有代码实际上就是被置于\texttt{tcolorbox}环境之中。

下面我们仅就最常用的\texttt{tabularx}给出示例,其余内容见给出的网络链接。

\begin{codeout}
\begin{table}[H]
\caption{tabularx示例}
\begin{center}
\begin{tabularx}{0.7\linewidth}{rp{0.2\linewidth}p{0.2\linewidth}}
 \toprule
 左对齐列 & 中间对齐列 & 右对齐列 \\
 \midrule
 abcde  & o & edcba \\
 123   & 0 & 321 \\
 \bottomrule
\end{tabularx}
\end{center}
\end{table}
\end{codeout}

\texttt{tabular}和\texttt{tabularx}是可以嵌套使用的。

\begin{codeout}
\begin{table}[H]
 \centering
 \caption{嵌套tabular}
 \begin{tabularx}{0.8\linewidth}{p{0.3\linewidth}X}
 \toprule
 分类 & 描述 \\
 \midrule
 类型一 &  \begin{tabular}[t]{ll}
           子类A & ab   \\
           子类B & sadf \\ 
           子类C & adf  \\ 
          \end{tabular}   \\
 类型二  & 较为简单的情况  \\
 \bottomrule
 \end{tabularx} 
 \label{tab:loopbaublar} 
\end{table}
\end{codeout}

\subsection{插图}
参考阅读:\url{https://en.wikibooks.org/wiki/LaTeX/Floats,_Figures_and_Captions}

如前所述,插图是另一类浮动体环境——\texttt{figure}。如:
\begin{codeout}
\begin{figure}[H]
  \begin{center}
  \includegraphics[width=.3\linewidth]{ctanlion} 
  \caption{\LaTeX{}的吉祥物}
  \label{fig:lion}
  \end{center}
\end{figure}
\end{codeout}

\LaTeX{}的吉祥物是一只可爱的狮子,如图\ref{fig:lion}所示,由Duane Bibby创作。插图默认的文件为\texttt{.eps}格式,也可以是\texttt{.jpg}或\texttt{.pdf}等。关于插图的更多内容参见网络链接。

\subsection{参考文献}
命令\verb!\bibliography!用来引用某个\texttt{bib}文件,如:
\begin{code}
\bibliography{./Reference/Reference.bib}
\end{code}

在\texttt{bib}文件中的所有文献都有一个\texttt{Bibtexkey}项代表该文献,在\texttt{tex}文件中用\verb!\cite!命令即可通过该项值引用文献,如:

\begin{code}
文献\cite{Wu2010}指出了如下观点:
\end{code}

\subsection{目录}
插入目录只需一条命令,位于\verb!\maketitle!之后:
\begin{code}
  \tableofcontents
\end{code}
\section{自定义}
\subsection{模板}
\LaTeX{}最吸人的地方就在于我们可以凭借定义好的模板,将全部精力放在写作中,而完全不必担心会出现格式错误。使用模板的方式很简单,即在文档类型中使用模板名即可,如:
\begin{code}
\documentclass{hitec}
\end{code}
模板文件的扩展名为\texttt{.cls},要使用\texttt{TeXLive}套件之外的模板,需要将该文件置于\TeX{}系统可搜索的目录中,然后刷新\TeX{}数据库,详见\ref{subsec:installtemplate}节。

\subsection{命令}
为了简化某些固定的输入内容,可以使用\verb!\newcommand!自定义命令。比如,我们希望在公式中固定单位\verb! MPa!。注意,我们要求单位之前留有半个空格,以和数字区分开。但如果我们总是手写输入,难免会在格式上出错。现在自定义一个命令\verb!\MPa!就可以避免这个麻烦。
\begin{codeout}
\newcommand{\MPa}{\ \mathrm{MPa}}
$5\MPa$
\end{codeout}

自定义命令还可以带参数,如:
\begin{codeout}
\newcommand{\emphtext}[1]{\textcolor{red}{\texttt{#1}}}
\emphtext{这是一句重要的废话。}
\end{codeout}

\subsection{环境}
当我们要将特别定制的样式应用于较多的内容时,将它们作为命令参数被作用就显得不够方便。此时,我们可以自定义环境。自定义环境格式为:
\begin{code}
 \newenvironment{环境名}{环境起始定义}{环境终止定义}
\end{code}

一般自定义环境是已有环境的扩展,比如:
\begin{codeout}
 \newenvironment{centertt}{
    \begin{center}
    \fangsong
    \color{blue}
 }
 {
    \end{center}
 } 

 \begin{centertt}
   居中的蓝色仿宋体环境
 \end{centertt} 

\end{codeout}
  
\subsection{自定义文件}
当我们自定义的命令或环境越来越多,放在正文中会影响正文的阅读。这时要么将这些定义放在模板中,要么单独放在一个\verb!tex!文件中,比如\verb!symbols.tex!,然后用一条命令就纳入到文档中了:
\begin{code}
  \newcommand{\integralx}[3]{\int_{#1}^{#2} #3 \mathrm{d}x }
\newcommand{\mafan}[3]{\sqrt{\sqrt{{#1}^2+{#2}^2}+{#3}^{4}}}
%%% Local Variables:
%%% mode: latex
%%% TeX-master: "latexwritting"
%%% End:

\end{code}
通常我们只会专注于某一领域的写作,所以随着这个文件的内容越来越多,我们将它放在其它文档中重复使用,以提高我们的写作效率。


\section{文档组织}

\subsection{安装模板}
\label{subsec:installtemplate}
在使用某个模板前,需要将模板加入到\texttt{texlive}系统中。这样编译文档时才能找得到这个模板。把模板装到\texttt{texlive}中统共分两步:
\begin{itemize}
\item 将模板文件\texttt{.cls}或\texttt{.sty}拷贝到模板目录。模板目录在\texttt{texlive}安装目录中的子文件夹\texttt{/texmf-local}。模板拷贝到这个文件夹中的任何子目录即可,但为了便于管理,还是再细致查找到合适的子目录再拷贝为好,比如:\texttt{/texmf-local/tex/latext/local}。
\item 在字符界面(或称命令行)中\texttt{texhash}刷新\texttt{texlive}数据库。
\end{itemize}

注意,若在\texttt{LINUX}系统中需要获得管理员权限后运行刷新命令,我们可以看到刷新数据库的路径显示出来:
\begin{cmd}
  > sudo texhash
  texhash: Updating /usr/local/texlive/2015/texmf-config/ls-R... 
  texhash: Updating /usr/local/texlive/2015/texmf-dist/ls-R... 
  texhash: Updating /usr/local/texlive/2015/texmf-var/ls-R... 
  texhash: Updating /usr/local/texlive/texmf-local/ls-R... 
  texhash: Done.
\end{cmd}


\subsection{文件分类存放}
通常我们的文档会越写越大,插图、子文档越来越多。为了便于我们的文档管理,需要分类存放文件。如插图放在子文件夹\texttt{figure},子文档放在子文件夹\texttt{body}中,而仅令将主文档(一般命名为\texttt{main.tex})放在一级目录中。在我们技术报告中,经常会反复使用一些数学符号,这时最好的处理方式就是将它们定义为命令,而这些命令定义最好放在一个专门的文件中,我们约定这个文件为\texttt{symbols.tex},也放在文件夹\texttt{body}中。


\section{结语}
基于以上介绍,你已初步了解了撰写一篇文档需要的大部分\LaTeX{}排版命令。真正掌握它们还需要大量的练习。除了网络和电子书籍,请教团队中的师兄和老师也是非常好的捷径。

本文即是入门介绍,又是一个示例,将在未来的工作中不断补充和丰富,同时也欢迎你也加入到这个文档的写作之中。

\section*{待解决问题}
\begin{table}[H]
  \centering
  \caption{文档日志}
  \label{tab:asdf}
  \begin{tabularx}{1.0\linewidth}{Xccc}
    \toprule
    描述   & 提出日期 & 解决日期 & 附注 \\
    \midrule
   建立文档 & 2015.10.03 &  2015.10.03 &  \\
   减小表前表后空白 & 2015.10.03 &       &  \\
   tabularx对齐和设置固度 & 2015.10.03 & 2015.10.03  &  \\
   设计了代码环境 & 2015.10.14 & 2015.10.14 & \\
   代码环境显示行号 & 2015.10.14 &    & \\
    \bottomrule
  \end{tabularx}
\end{table}


\end{document}

%%% Local Variables:
%%% mode: latex
%%% TeX-master: t
%%% End:
