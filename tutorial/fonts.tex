\section{字体}

我们在写作时为了强调某些内容,或纯粹为了避免单调,会适度地转变字体。在排
版中,字体其实是一个含义丰富的术语。当我们在说一种字体时,实际上是一个属
性集合,它包括:

\begin{codeout}
  \begin{description}
  \item[大小] 比如:{\zihao{3}字号三}、{\zihao{5}字号五}、{\zihao{7}字号
      七}。
  \item[字形] 比如:粗体{\textbf{bfseries}}、斜体{\textsl{slant}}、打印
    体{\texttt{italic}}。
  \item[字族] 比如:\songti{宋体}、\heiti{黑体}、\fangsong{仿宋}。
  \end{description}
\end{codeout}

其它还有一个属性“字族”,因很少涉及,这里暂略去。关于字体还有更丰富的命
令,暂时也不详谈。

改变字体一般有两种方式:一种是改变所选择的文字,称为“区域命令”;另一种
是改变命令声明以后的字体,称为“声明命令”或“模式命令”。

也许你会在有的地方看过命令,如\verb!\bf!,\verb!\md!,\verb!it!,
\verb!\sl!,\verb!\sc!,\verb!\sf!,\verb!\tt!,或\verb!\rm!等。这些命令
已经是陈旧命令,不提倡再使用。原因这些命令不允许复合作用,如:

\begin{codeout}
  \it{\bf{ it and bf format}}
\end{codeout}

显然,\verb!\it!命令已经被\verb!\bf!屏蔽了。1994年发布了\LaTeXe{}取代了老
版本的\LaTeX{}的2.09版,使用了新的命令风格\verb!\textxx!,如:

\begin{codeout}
  \textit{\textbf{Never mind the rain}}
\end{codeout}

改进后的字体命令就可以嵌套作用了。

字体声明命令见表\ref{tbl:fontdeclaration}。字体声明命令还可以作为环境使
用。表中命令嵌套使用对临时设置字体时很重要。

\begin{table}[H]
  \centering
  \caption{字体改变声明}
  \label{tbl:fontdeclaration}
  \begin{tabular}{lll}
    \toprule
    \textbf{Declaration} & \textbf{Example Input} & \textbf{Corresponding output} \\ 
    \midrule
    \verb+\rmfamily+ & \verb+\rmfamily roman text+ & {\rmfamily roman text }\\
    \verb+\sffamily+ & \verb+\sffamily sans serif text+ & {\sffamily sans serif text }\\ 
    \verb+\ttfamily+ & \verb+\ttfamily typewritter text+ & {\ttfamily typewritter text}\\
    \verb+\mdseries+ & \verb+\mdseries medium text+ & {\mdseries medium text}\\ 
    \verb+\upshape+  & \verb+\upshape upright text+ & {\upshape upright text}\\ 
    \verb+\itshape+ & \verb+\itshape italic text+ & {\itshape italic text}\\ 
    \verb+\slshape+ & \verb+\slshape slant+ & {\em emphasized text} \\
    \verb+\normalfont+ & \verb+\normalfont default text+ & {\normalfont
                                                           default font}\\
    \bottomrule
  \end{tabular}
\end{table}


%%% Local Variables:
%%% mode: latex
%%% TeX-master: "latexwritting"
%%% End:
