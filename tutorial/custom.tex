\section{自定义}

\subsection{模板}

\LaTeX{}最吸人的地方就在于我们可以凭借定义好的模板,将全部精力放在写作中,
而完全不必担心会出现格式错误。使用模板的方式很简单,即在文档类型中使用模
板名即可,如:

\begin{code}
  \documentclass{hitec}
\end{code}

模板文件的扩展名为\texttt{.cls},要使用\texttt{TeXLive}套件之外的模板,需
要将该文件置于\TeX{}系统可搜索的目录中,然后刷新\TeX{}数据库,详
见\ref{subsec:installtemplate}节。

\subsection{命令}

为了简化某些固定的输入内容,可以使用\verb!\newcommand! 自定义命令。比如,
我们希望在公式中固定单位\verb! MPa!。注意,我们要求单位之前留有半个空格,
以和数字区分开。但如果我们总是手写输入,难免会在格式上出错。现在自定义一
个命令\verb!\MPa!就可以避免这个麻烦。

\begin{codeout}
  \newcommand{\MPa}{\ \mathrm{MPa}}
  $5\MPa$
\end{codeout}

自定义命令还可以带参数,如:

\begin{codeout}
  \newcommand{\emphtext}[1]{\textcolor{red}{\texttt{#1}}}
  \emphtext{这是一句重要的废话。}
\end{codeout}

\subsection{环境}

当我们要将特别定制的样式应用于较多的内容时,将它们作为命令参数被作用就显
得不够方便。此时,我们可以自定义环境。自定义环境格式为:

\begin{code}
  \newenvironment{环境名}{环境起始定义}{环境终止定义}
\end{code}

一般自定义环境是已有环境的扩展,比如:

\begin{codeout}
  \newenvironment{centertt}{%
      \begin{center} \fangsong \color{blue}
    }%
    {%
      \end{center}
    }%

    \begin{centertt}
      居中的蓝色仿宋体环境
    \end{centertt}

\end{codeout}
  
\subsection{自定义文件}

当我们自定义的命令或环境越来越多,放在正文中会影响正文的阅读。这时要么将
这些定义放在模板中,要么单独放在一个\verb!tex!文件中,比
如\verb!symbols.tex!,然后用一条命令就纳入到文档中了:

\begin{code}
  \newcommand{\integralx}[3]{\int_{#1}^{#2} #3 \mathrm{d}x }
\newcommand{\mafan}[3]{\sqrt{\sqrt{{#1}^2+{#2}^2}+{#3}^{4}}}
%%% Local Variables:
%%% mode: latex
%%% TeX-master: "latexwritting"
%%% End:

\end{code}

通常我们只会专注于某一领域的写作,所以随着这个文件的内容越来越多,我们将
它放在其它文档中重复使用,以提高我们的写作效率。


\section{文档组织}

\subsection{安装模板}
\label{subsec:installtemplate}

在使用某个模板前,需要将模板加入到\texttt{texlive}系统中。这样编译文档时
才能找得到这个模板。把模板装到\texttt{texlive}中统共分两步:

\begin{itemize}
\item 将模板文件\texttt{.cls}或\texttt{.sty}拷贝到模板目录。模板目录
  在\texttt{texlive}安装目录中的子文件夹\texttt{/texmf-local}。模板拷贝到
  这个文件夹中的任何子目录即可,但为了便于管理,还是再细致查找到合适的子
  目录再拷贝为好,比如:

  \texttt{/texmf-local/tex/latext/local}。
\item 在字符界面(或称命令行)中\texttt{texhash}刷新\texttt{texlive}数据
  库。
\end{itemize}

注意,若在\texttt{LINUX}系统中需要获得管理员权限后运行刷新命令,我们可以
看到刷新数据库的路径显示出来:

\begin{cmd}
  > sudo texhash
  texhash: Updating /usr/local/texlive/2015/texmf-config/ls-R...
  texhash: Updating /usr/local/texlive/2015/texmf-dist/ls-R...
  texhash: Updating /usr/local/texlive/2015/texmf-var/ls-R...
  texhash: Updating /usr/local/texlive/texmf-local/ls-R...
  texhash: Done.
\end{cmd}


\subsection{文件分类存放}

通常我们的文档会越写越大,插图、子文档越来越多。为了便于我们的文档管理,
需要分类存放文件。如插图放在子文件夹\texttt{figure},子文档放在子文件
夹\texttt{body}中,而仅令将主文档(一般命名为\texttt{main.tex})放在一级
目录中。在我们技术报告中,经常会反复使用一些数学符号,这时最好的处理方式
就是将它们定义为命令,而这些命令定义最好放在一个专门的文件中,我们约定这
个文件为\texttt{symbols.tex},也放在文件夹\texttt{body}中。


%%% Local Variables:
%%% mode: latex
%%% TeX-master: "latexwritting"
%%% End:
