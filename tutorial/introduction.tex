\section{前言}

在动笔之前,我们冒昧地对读者做了一个大胆的假设:\emph{您刚刚听说\LaTeX{},
  虽然所知不多,但已决意学习。}之所以需要这样的假设,纯粹是因为我们想暂时
避开一个非常棘手的问题:\emph{为什么要学习\LaTeX{}}。本文意在简介,只是扮
演一个门童的角色,为您说明进入\LaTeX{}殿堂的门径,无意以偏盖全。倘若被某
位行家看到错误之处,还望指正。

% 本文假定你已经看过某人在你面前演示过\LaTeX{}的编译过程,教会了你如何安
% 装\texttt{texlive}和如何使用某款撰写和编译\LaTeX{}的集成环境,
% 如\texttt{TeXMaker}。

% 本文包括\LaTeX{}基本命令和环境、自定义命令和环境、自定义符号的方法。为了
% 实现合作写作,我们还做了一些关于文件组织的约定。在模板的使用过程中,我们
% 会发现一些实际需求,比如老师的批注、对其批注的引用等,对于需求我们将定义
% 一些命令或环境,这样有助于我们提升交流的质量。在\LaTeX{}的介绍过程中,文
% 中还提供了相关内容的来源。追溯这些来源,我们就可以慢慢详细了解\LaTeX{}知
% 识。

\subsection{\LaTeX{}的历史}

在不了解\LaTeX{}的背景知识之前可以学习它吗?当然可以。但考虑到每个人的学
习习惯并不一样,这里提供一些链接,让好奇心更强的读者在学习之前大快朵颐一
番。

\begin{itemize}
\item 中文:\url{http://www.ctex.org/documents/shredder/tex_frame.html}
\item 英文:\url{https://en.wikipedia.org/wiki/LaTeX}
\end{itemize}

第一次就能读准它很重要:\LaTeX{}——“雷泰赫”。

\subsection{自学资料}

如果你没兴趣看本文的解释,情愿自己去学习权威教程或获取更丰富的资源。那就
给你推荐几个好去处。

\begin{itemize}
\item \url{https://en.wikibooks.org/wiki/LaTeX}
\item \url{http://www.tug.org/twg/mactex/tutorials}
\item \LaTeX{} for Complete Novices: 本地文档,执行命令打开该文档。
  \begin{cmd} texdoc dickimaw-novices
  \end{cmd}
 \item \LaTeXe{}: An unofficial reference manual
  \begin{cmd} texdoc latex2e
 \end{cmd}
 \item The Not So Short Introduction to \LaTeXe{}
   \begin{cmd} texdoc lshort
   \end{cmd}
\item 刘海洋. \LaTeX{}入门,电子工业出版社,2013
\item TUG: \url{http://www.tug.org},\TeX{}用户群(\TeX Users Group,
  TUG)网站,全世界\TeX{}用户的组织。这里可以下载到\texttt{TeXLive CD}。
\item CTAN: \url{http://www.ctan.org},(Comprehensive TeX Archive
  Network,CTAN),你想要的所有文档、模板都在这里。再强调一下,是所有的都
  在这里。
\item LaTeX: \url{http://www.latex-project.org/},\LaTeX{}官方网站。
\item Stackexchange:\url{http://stackexchange.com},专业的技术交流网站。
\item \LaTeX{}工作室:\url{http://www.latexstudio.cn},提供国内稀有高水准排版
  服务。还有一个QQ社区:91940767。
\end{itemize}

%%% Local Variables:
%%% mode: latex
%%% TeX-master: "latexwritting"
%%% End:
