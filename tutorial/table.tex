\section{表格}
参考阅读:\url{https://en.wikibooks.org/wiki/LaTeX/Tables}

表格和后面要讲的插图,在\LaTeX{}中称为\texttt{浮动体}。各类表格环境声明
于浮动体\texttt{table}之中。浮动体的位置通过选项为控制,一般我们为了使表
格处于期望的当前位置,需要引用一个宏包\texttt{float},然后使用选项
\texttt{[H]}。要引用表格,需要在浮动体中用命令\verb!\label!声明该浮动体
的名字,再用\verb!\ref!引用即可。

\begin{codeout}
\begin{table}[H]
\caption{tabular示例}
\label{table:first}
\begin{center}
%\begin{tabular}{rcl}
\begin{tabular}{>{\raggedleft}p{0.2\linewidth}
  >{\centering}p{0.2\linewidth} p{0.2\linewidth}}
  \toprule
  左对齐列 & 中间对齐列 & 右对齐列 \\
  \hline
  abcde    & o          & edcba    \\
  123      & 0          & 321      \\
\bottomrule
\end{tabular}
\end{center}
\end{table}

计算结果见表\ref{table:first}。

\end{codeout}

\LaTeX{}中的表格仅有\texttt{tabbing}和\texttt{tabluar}两种环境,为了丰富
和提升表格的功能,大量表格类宏包被开发出来,如可定制宽度的
\texttt{tabularx}、可跨页的\texttt{longtable}、具有更多丰富功能的
\texttt{tabu}、以及彩色盒子\texttt{tcolorbox}等。本文中的所有代码实际上
就是被置于\texttt{tcolorbox}环境之中。

下面我们仅就最常用的\texttt{tabularx}给出示例,其余内容见给出的网络链接。

\begin{codeout}
\begin{table}[H]
\caption{tabularx示例}
\begin{center}
\begin{tabularx}{0.7\linewidth}{rp{0.2\linewidth}p{0.2\linewidth}}
  \toprule
  左对齐列 & 中间对齐列 & 右对齐列 \\
  \midrule
  abcde & o & edcba \\
  123 & 0 & 321 \\
  \bottomrule
\end{tabularx}
\end{center}
\end{table}
\end{codeout}

\texttt{tabular}和\texttt{tabularx}是可以嵌套使用的。

\begin{codeout}
\begin{table}[H] \centering
 \caption{嵌套tabular}
 \begin{tabularx}{0.8\linewidth}{p{0.3\linewidth}X}
   \toprule
   分类 & 描述 \\
   \midrule
   类型一 & \begin{tabular}[t]{ll}子类A & ab \\子类B & sadf \\
子类C & adf \\
            \end{tabular} \\
   类型二 & 较为简单的情况 \\
   \bottomrule
 \end{tabularx}
 \label{tab:loopbaublar}
\end{table}
% \end{codeout}

%%% Local Variables:
%%% mode: latex
%%% TeX-master: "latexwritting"
%%% End:
