\section{文档结构}

用\LaTeX{}写作统共分两步:
\begin{enumerate}
\item 撰写\texttt{tex}文档;
\item 编译文档并转为特定的阅读格式,一般是\texttt{pdf}文档。
\end{enumerate}

有人会问:这不和编程一样吗?是的,用\LaTeX{}撰写文档基本上就是一个“用文字
编程”的过程。

我们要做的其实只是撰写一些后缀为\textit{tex}的文本文档,其余的事都交给程
序去做。首先介绍\textit{tex}文档的基本结构。\LaTeX{}文档包
括\uline{导言}和\uline{主体}两部分。

导言部分用于声明文档的\uline{类型(class)}和引用
的\uline{宏包(package)}。\uline{类型}决定了文档编译后所套用的模板、为文档
撰写准备的一些变量设置。\uline{宏包}提供了很多命令和环境供我们使用。

主体结构由一个环境构成\verb!\begin{document} ... \end{document}!。在这两
个命令中间便是你撰写文档内容的地方。文档中\verb!%!之后本行内容不被编译,
换言之,\verb!%!是\uline{注释符}。所有\LaTeX{}命令都是用\verb!\!开头的,
它的中文名字叫\uline{反斜杠}。

  
\begin{code}
  % -----------------导言区
  \documentclass{hitec}   %指定文档所套用的模板
  \usepackage[UTF8]{ctex} %引用宏包并加注选项
  \usepackage{amsmath}    %引用宏包
  \usepackage{tcolorbox}
  \tcbuselibrary{listings}
  % -----------------主体
  \begin{document}
  % 您的大作
  \end{document}
\end{code}

当你写完了文档,就可以编译文档了,比如文档名为\texttt{main.tex}。如果在命
令行中操作,是这样的:

\begin{cmd}
  > latex main.tex
\end{cmd}

但我们多数文档会含有中文,此时应使用\texttt{xelatex}编译:

\begin{cmd}
  > xelatex main.tex
\end{cmd}

前面说了,\texttt{xelatex},或者\texttt{pdflatex}就是\emph{编译引擎},其实就是
个可执行程序。如果程序在编译中发现文档中有语法错误,就会报错。错误信息显
示在屏幕上,也会写在后缀为\textit{log}的文件中。

然而实际上我们很少在命令行中输入命令编译,多数是在一个集成环境中用快捷完
成编译。比如在\texttt{TeXMaker}中的\texttt{F1}键,在\texttt{Emacs
AUCTex}中的\texttt{C-c C-c}。总之,在撰写\LaTeX{}文档时,有一款高效的集
成环境是非常必要的。

%%% Local Variables:
%%% mode: latex
%%% TeX-master: "latexwritting"
%%% End:
